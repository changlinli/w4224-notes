\documentclass{article}

\usepackage{amsmath, amssymb, amsfonts, amsthm, booktabs, verbatim, mathtools, listings, xcolor}

%\numberwithin{equation}{section}
\newtheorem{theorem}{Theorem}
\newtheorem{lemma}{Lemma}
\newtheorem{proposition}{Proposition}
\newtheorem{definition}{Definition}
\newtheorem{problem}{Problem}

% Surrounding angular brackets
\newcommand{\surang}[1]{\langle #1 \rangle}
\makeatletter
\newcommand{\@giventhatstar}[2]{#1\,\middle|\,#2}
\newcommand{\@giventhatnostar}[3][]{#1(#2\,#1|\,#3#1)}
\newcommand{\giventhat}{\@ifstar\@giventhatstar\@giventhatnostar}
\makeatother
\newcommand{\probof}[1]{\text{Pr}\left( #1 \right)}
\newcommand{\pdens}[1]{\text{p}\left( #1 \right)}
\newcommand{\variance}[1]{\text{Var}\left( #1 \right)}

\lstset{ %
  language=R,                     % the language of the code
  basicstyle=\footnotesize,       % the size of the fonts that are used for the code
  numbers=left,                   % where to put the line-numbers
  numberstyle=\tiny\color{gray},  % the style that is used for the line-numbers
  stepnumber=1,                   % the step between two line-numbers. If it's 1, each line
                                  % will be numbered
  numbersep=5pt,                  % how far the line-numbers are from the code
  backgroundcolor=\color{white},  % choose the background color. You must add \usepackage{color}
  showspaces=false,               % show spaces adding particular underscores
  showstringspaces=false,         % underline spaces within strings
  showtabs=false,                 % show tabs within strings adding particular underscores
  frame=single,                   % adds a frame around the code
  rulecolor=\color{black},        % if not set, the frame-color may be changed on line-breaks within not-black text (e.g. commens (green here))
  tabsize=2,                      % sets default tabsize to 2 spaces
  captionpos=b,                   % sets the caption-position to bottom
  breaklines=true,                % sets automatic line breaking
  breakatwhitespace=false,        % sets if automatic breaks should only happen at whitespace
  title=\lstname,                 % show the filename of files included with \lstinputlisting;
                                  % also try caption instead of title
  keywordstyle=\color{blue},      % keyword style
  commentstyle=\color{red},   % comment style
  stringstyle=\color{black},      % string literal style
  escapeinside={\%*}{*)},         % if you want to add a comment within your code
  morekeywords={*,...}            % if you want to add more keywords to the set
} 

\begin{document}
Note an errata in the book; for Equation~8.2, we have that $\pdens{\giventhat*{\theta}{x, y_{\text{obs}}, I}} = \pdens{\giventhat*{\theta}{x}} \int \int \pdens{\giventhat*{\phi}{x, \theta}} \pdens{\giventhat*{y}{x, \theta}} \pdens{\giventhat*{I}{x, y, \phi}} dy_\text{mis} d\phi$ when we should have a proportionality symbol instead of equality.

\begin{theorem}
	Given the condition of missing at random and the condition of distinct parameters we have ignorable.
\end{theorem}
\begin{proof}
	\begin{align}
		\pdens{\giventhat*{\theta}{y_{\text{obs}}, I}} &\propto \pdens{\theta} \int \int \pdens{\giventhat*{\phi}{\theta}} \pdens{\giventhat*{y}{\theta}} \pdens{\giventhat*{I}{y, \phi}} dy_\text{mis} d\phi\\
		&= \pdens{\theta} \int \int \pdens{\phi} \pdens{\giventhat*{y}{\theta}} \pdens{\giventhat*{I}{y_{\text{obs}}, \phi}} dy_{\text{mis}} d\phi\\
		&= \pdens{\theta} \int \pdens{\theta} \pdens{\giventhat*{I}{y_\text{obs}, \phi}} \left( \int \pdens{\giventhat*{y}{\theta}} dy_\text{mis} \right) d\theta\\
		&= \pdens{\theta} \pdens{\giventhat*{y_\text{obs}}{\theta}} \int \pdens{\phi} \pdens{\giventhat*{I}{y_\text{obs}, \phi}} d\phi \\
		&\propto \pdens{\theta} \pdens{\giventhat*{y_\text{obs}}{\theta}}\\
		&\propto \pdens{\giventhat*{\theta}{y_\text{obs}}}
	\end{align}
\end{proof}

Missing at random is violated when the sampling depends on the data value.

For homework problem 1 the key term is negative hypergeometric. The negative hypergeometric distribution is to the hypergeometric distribution what the negative binomial is to the binomial.

The negative hypergeometric is to the negative binomial as the hypergeometric is to the binomial.

We do the classic bound above by an expression to get that the posterior is proper.

The probability that the next fish is tagged.

\begin{align*}
	\probof{\giventhat*{\tilde{y} = 1}{y}} &= \sum _{N = 170} ^\infty \probof{\giventhat*{\tilde{y} = 1}{N}} \pdens{\giventhat*{N}{y}}\\
	&= \sum _{N = 170} ^\infty \left( \frac{80}{N - 90} \right) \pdens{\giventhat*{N}{y}}\\
	&\approx \frac{20}{90} \\
	&= 0.22
\end{align*}

\end{document}
